\chapter{Аналитический раздел}

\section{Постановка задачи}

В соответствии с заданием на курсовую работу необходимо разработать загружаемый модуля ядра Linux для отключения графического ускорителя при подключении USB--устройства. Для решения данной задачи необходимо:

\begin{itemize}
	\item проанализировать методы обработки событий, возникающих при взаимодействии с USB--устройствами;
	\item проанализировать структуры и функции ядра, предоставляющие информацию о USB--устройствах;
	\item разработать алгоритмы и структуру программного обеспечения;
	\item реализовать программное обеспечение;
	\item проанализировать разработанное программное обеспечение.
\end{itemize}

\section{Обработка событий от USB--устройств}

Для обработки событий, возникающих при работе с USB--устройствами, например, таких как подключение или отключение устройства, необходимо узнать о возникновении события и выполнить необходимую обработку после возникновения события.

Далее будут рассмотрены существующие различные подходы к определению возникновения событий от USB--устройств и выбран наиболее подходящий для реализации.

\newpage
\subsection{Уведомители}

Ядро Linux содержит механизм, называемый <<уведомителями>> (\texttt{notifiers}) или <<цепочками уведомлений>> (\texttt{notifiers chains}), который позволяет различным подсистемам подписываться на асинхронные события от других подсистем. Цепочки уведомлений в настоящее время активно используется в ядре; существуют цепочки для событий \texttt{hotplug} памяти, изменения политики частоты процессора, события \texttt{USB hotplug}, загрузки и выгрузки модулей \cite{notifications}.

В листинге \ref{lst:notifier_block} представлена структура \texttt{notifier\_block} \cite{notifier_block}.

\begin{lstlisting}[
	caption={Структура \texttt{notifier\_block}},
	label={lst:notifier_block},
	style={c},
	]
struct notifier_block {
	notifier_fn_t notifier_call;
	struct notifier_block __rcu *next;
	int priority;
};
\end{lstlisting}

Данная структура описана в \texttt{/include/linux/notifier.h}. Она содержит указатель на функцию--обработчик уведомления (\texttt{notifier\_call}), указатель на следующий уведомитель (\texttt{next}) и приоритет уведомителя (\texttt{priority}). Уведомители с более высоким значением приоритета выполняются первее.

В листинге \ref{lst:notifier_fn_t} представлена сигнатура функии \texttt{notifier\_call}.

\begin{lstlisting}[
	caption={Тип \texttt{notifier\_fn\_t}},
	label={lst:notifier_fn_t},
	style={c},
	]
typedef	int (*notifier_fn_t)(struct notifier_block *nb, unsigned long action, void *data);
\end{lstlisting}

Сигнатура содержит указатель на уведомитель (\texttt{nb}), действие, при котором срабатывает функция (\texttt{action}) и данные, которые передаются от действия в обработчик (\texttt{data}).

\newpage
Для регистрации уведомителя для USB--портов используются функции регистрации и удаления уведомителя, представленные в листинге \ref{lst:usb_notifiers}.

\begin{lstlisting}[
	caption={Уведомители на USB--портах},
	label={lst:usb_notifiers},
	style={c},
	]
/* Events from the usb core */
#define USB_DEVICE_ADD		    0x0001
#define USB_DEVICE_REMOVE	0x0002
#define USB_BUS_ADD		           0x0003
#define USB_BUS_REMOVE		   0x0004
extern void usb_register_notify(struct notifier_block *nb);
extern void usb_unregister_notify(struct notifier_block *nb);
\end{lstlisting}

Прототипы и константы для действий описаны в файле\\ \texttt{/include/linux/notifier.h}, а реализации функций --- в файле\\ \texttt{/drivers/usb/core/notify.c}. Соответственно действие \texttt{USB\_DEVICE\_ADD} означает подключение нового устройства, а \texttt{USB\_DEVICE\_REMOVE} --- удаление.

Особенности \texttt{уведомителей}:

\begin{itemize}
	\item возможность привязки своего обработчика к событию;
	\item возможность добавления более чем одного обработчика событий;
	\item возможность использования интерфейса в загружаемом модуле ядра;
\end{itemize}

\subsection{usbmon}

\texttt{usbmon} \cite{usbmon} --- средство ядра Linux, используемое для сбора информации о событиях, произошедших на устройствах ввода--вывода, подключенных через USB.

\texttt{usbmon} предоставляет информацию о запросах, сделанных дравйверами устройств к драйверам хост--контроллера (HCD). Если драйвера хост--контроллера неисправны, то данные, предоставленные \texttt{usbmon}, могут не соответствовать действительным переданным данным.

В настоящее время реализованы два программных интерфейса для взаимодействия с \texttt{usbmon}: текстовый и двоичный. Двоичный интерфейс доступен через символьное устройство в пространстве имен \texttt{/dev}. Текстовый интерфейс устарел, но сохраняется для совместимости.

В листинге \ref{lst:usbmon} представлена структура ответа, полученного после события, случившегося на USB--устройстве (например, подключение к компьютеру).

\begin{lstlisting}[
	caption={Структура \texttt{usbmon\_packet}},
	label={lst:usbmon},
	style={c},
	]
struct usbmon_packet {
	u64 id;			/*  0: URB ID - from submission to callback */
	unsigned char type;	/*  8: Same as text; extensible. */
	unsigned char xfer_type; /*    ISO (0), Intr, Control, Bulk (3) */
	unsigned char epnum;	/*     Endpoint number and transfer direction */
	unsigned char devnum;	/*     Device address */
	u16 busnum;		/* 12: Bus number */
	char flag_setup;	/* 14: Same as text */
	char flag_data;		/* 15: Same as text; Binary zero is OK. */
	s64 ts_sec;		/* 16: gettimeofday */
	s32 ts_usec;		/* 24: gettimeofday */
	int status;		/* 28: */
	unsigned int length;	/* 32: Length of data (submitted or actual) */
	unsigned int len_cap;	/* 36: Delivered length */
	union {			/* 40: */
		unsigned char setup[SETUP_LEN];	/* Only for Control S-type */
		struct iso_rec {		/* Only for ISO */
			int error_count;
			int numdesc;
		} iso;
	} s;
	int interval;		/* 48: Only for Interrupt and ISO */
	int start_frame;	/* 52: For ISO */
	unsigned int xfer_flags; /* 56: copy of URB's transfer_flags */
	unsigned int ndesc;	/* 60: Actual number of ISO descriptors */
};				/* 64 total length */
\end{lstlisting}

Особенности \texttt{usbmon}:

\begin{itemize}
	\item возможность просматривать собранную информацию через специальное ПО (например, \texttt{Wireshark});
	\item возможность отслеживать события на одном порте USB или на всех сразу;
	\item отсутствие возможности вызова обработчика при возникновении определенного события.
\end{itemize}

\texttt{usbmon} позволяет отслеживать события, но не позволяет реагировать на них без программной доработки для реализации обработчика.

\subsection{udevadm}

\texttt{udevadm} \cite{udevadm} --- инструмент для управления устройствами \texttt{udev}. Структура \texttt{udev} описана в библиотеке \texttt{libudev} \cite{libudev}, которая не является системной библиотекой Linux. В данной библиотеке представлен программный интерфейс для мониторинга и взаимодействия с локальными устройствами.

При помощи \texttt{udevadm} можно получить полную информацию об устройстве, полученную из его представления в \texttt{sysfs}, чтобы создать корректные правила и обработчики событий для устройства. Кроме того можно получить список событий для устройства, установить наблюдение за ним.

В листинге \ref{lst:udevadm} представлен пример правила обработки событий, задаваемого с помощью \texttt{udevadm}.

\begin{lstlisting}[
	caption={Правила \texttt{udevadm}},
	label={lst:udevadm},
	style={c},
	]
/* rules file */
SUBSYSTEM=="usb", ACTION=="add", ENV{DEVTYPE}=="usb_device",  RUN+="/bin/device_added.sh"
SUBSYSTEM=="usb", ACTION=="remove", ENV{DEVTYPE}=="usb_device", RUN+="/bin/device_removed.sh"

/* device_added.sh */
#!/bin/bash
echo "USB device added at $(date)" >>/tmp/scripts.log

/* device_removed.sh */
#!/bin/bash
echo "USB device removed  at $(date)" >>/tmp/scripts.log
\end{lstlisting}

Особенности \texttt{udevadm}:

\begin{itemize}
	\item возможность привязки своего обработчика к событию;
	\item невозможность использования интерфейса в ядре Linux;
\end{itemize}

\section{USB--устройства в ядре Linux}

\subsection{Структура \texttt{usb\_device}}

Для хранения информации о USB--устройстве в ядре используется структура \texttt{usb\_device}, описанная в \texttt{/inlclude/linux/usb.h} \cite{usb_device}.

Структура \texttt{usb\_device} предствалена в листинге \ref{lst:usb_device}.

\begin{lstlisting}[
	caption={Структура \texttt{usb\_device}},
	label={lst:usb_device},
	style={c},
	]
struct usb_device {
	int		devnum;
	char		devpath[16];
	u32		route;
	enum usb_device_state	state;
	enum usb_device_speed	speed;
	unsigned int		rx_lanes;
	unsigned int		tx_lanes;
	enum usb_ssp_rate	ssp_rate;
	
	struct usb_tt	*tt;
	int		ttport;
	
	unsigned int toggle[2];
	
	struct usb_device *parent;
	struct usb_bus *bus;
	struct usb_host_endpoint ep0;
	
	struct device dev;
	
	struct usb_device_descriptor descriptor;
	struct usb_host_bos *bos;
	struct usb_host_config *config;
	
	struct usb_host_config *actconfig;
	struct usb_host_endpoint *ep_in[16];
	struct usb_host_endpoint *ep_out[16];
	
	char **rawdescriptors;
	
	unsigned short bus_mA;
	u8 portnum;
	u8 level;
	u8 devaddr;
	
	unsigned can_submit:1;
	unsigned persist_enabled:1;
	unsigned have_langid:1;
	unsigned authorized:1;
	unsigned authenticated:1;
	unsigned wusb:1;
	unsigned lpm_capable:1;
	unsigned usb2_hw_lpm_capable:1;
	unsigned usb2_hw_lpm_besl_capable:1;
	unsigned usb2_hw_lpm_enabled:1;
	unsigned usb2_hw_lpm_allowed:1;
	unsigned usb3_lpm_u1_enabled:1;
	unsigned usb3_lpm_u2_enabled:1;
	int string_langid;
	
	/* static strings from the device */
	char *product;
	char *manufacturer;
	char *serial;
	
	struct list_head filelist;
	
	int maxchild;
	
	u32 quirks;
	atomic_t urbnum;
	
	unsigned long active_duration;
	
#ifdef CONFIG_PM
	unsigned long connect_time;
	
	unsigned do_remote_wakeup:1;
	unsigned reset_resume:1;
	unsigned port_is_suspended:1;
#endif
	struct wusb_dev *wusb_dev;
	int slot_id;
	enum usb_device_removable removable;
	struct usb2_lpm_parameters l1_params;
	struct usb3_lpm_parameters u1_params;
	struct usb3_lpm_parameters u2_params;
	unsigned lpm_disable_count;
	
	u16 hub_delay;
	unsigned use_generic_driver:1;
};
\end{lstlisting}

Каждое USB--устройство должно соответствовать спецификации USB--IF \cite{usb_spec}, одним из требований которой является наличие идентификатора поставщика (\texttt{Vendor ID (VID)}) и идентификатор продукта (\texttt{Product ID (PID)}). Эти данные присутствуют в поле \texttt{descriptor} структуры \texttt{usb\_device}. В листинге \ref{lst:usb_device_descriptor} представлена структура дескриптора \texttt{usb\_device\_descriptor}, описанная в \texttt{/include/uapi/linux/usb/ch9.h}.

\begin{lstlisting}[
	caption={Структура \texttt{usb\_device\_descriptor}},
	label={lst:usb_device_descriptor},
	style={c},
	]
/* USB_DT_DEVICE: Device descriptor */
struct usb_device_descriptor {
	__u8  bLength;
	__u8  bDescriptorType;
	
	__le16 bcdUSB;
	__u8  bDeviceClass;
	__u8  bDeviceSubClass;
	__u8  bDeviceProtocol;
	__u8  bMaxPacketSize0;
	__le16 idVendor;
	__le16 idProduct;
	__le16 bcdDevice;
	__u8  iManufacturer;
	__u8  iProduct;
	__u8  iSerialNumber;
	__u8  bNumConfigurations;
} __attribute__ ((packed));

#define USB_DT_DEVICE_SIZE		18
\end{lstlisting}

\newpage
\subsection{Структура \texttt{usb\_device\_id}}

При подключении USB--устройства к компьютеру, оно идентифицируется и идентификационная информация записывается в структуру \texttt{usb\_device\_id} \cite{usb_device_id}.

Структура \texttt{usb\_device\_id} предствалена в листинге \ref{lst:usb_device_id}.

\begin{lstlisting}[
	caption={Структура \texttt{usb\_device\_id}},
	label={lst:usb_device_id},
	style={c},
	]
struct usb_device_id {
	/* which fields to match against? */
	__u16		match_flags;
	
	/* Used for product specific matches; range is inclusive */
	__u16		idVendor;
	__u16		idProduct;
	__u16		bcdDevice_lo;
	__u16		bcdDevice_hi;
	
	/* Used for device class matches */
	__u8		bDeviceClass;
	__u8		bDeviceSubClass;
	__u8		bDeviceProtocol;
	
	/* Used for interface class matches */
	__u8		bInterfaceClass;
	__u8		bInterfaceSubClass;
	__u8		bInterfaceProtocol;
	
	/* Used for vendor-specific interface matches */
	__u8		bInterfaceNumber;
	
	/* not matched against */
	kernel_ulong_t	driver_info
	__attribute__((aligned(sizeof(kernel_ulong_t))));
};
\end{lstlisting}

\newpage
\section{Запуск программ пользовательского пространства в пространстве ядра}

Для запуска программ пространства пользователя из пространства ядра используется \texttt{usermode-helper API}. Чтобы создать процесс из пространства пользователя необходимо указать имя исполняемого файла, аргументы, с которыми требуется запустить программу, и переменные окружения \cite{umhelper}.

В листинге \ref{lst:umh} представлена структура процесса, использующегося в \texttt{usermode-helper API} и сигнатура функции вызова \cite{umh}.

\begin{lstlisting}[
	caption={\texttt{usermode-helper API}},
	label={lst:umh},
	style={c},
	]
#define UMH_NO_WAIT	0	/* don't wait at all */
#define UMH_WAIT_EXEC	1	/* wait for the exec, but not the process */
#define UMH_WAIT_PROC	2	/* wait for the process to complete */
#define UMH_KILLABLE	4	/* wait for EXEC/PROC killable */

struct subprocess_info {
	struct work_struct work;
	struct completion *complete;
	const char *path;
	char **argv;
	char **envp;
	int wait;
	int retval;
	int (*init)(struct subprocess_info *info, struct cred *new);
	void (*cleanup)(struct subprocess_info *info);
	void *data;
} __randomize_layout;

extern int call_usermodehelper(const char *path, char **argv, char **envp, int wait);
\end{lstlisting}

\newpage
\section*{Выводы}

Были рассмотрены методы обработки событий, возникающих при взаимодействии с USB--устройствами. Среди рассмотренных методов был выбран механизм уведомителей, так как он позволяет привязать свой обработчик события, а также реализован на уровне ядра Linux. Были рассмотрены структуры и функции ядра для работы с уведомителями, а также особенности разработки загружаемых модулей ядра.