\chapter{Аналитический раздел}

\section{Постановка задачи}

В соответствии с заданием на курсовую работу необходимо разработать загружаемый модуля ядра Linux для отключения графического ускорителя при подключении USB--устройства. Для решения данной задачи необходимо:

\begin{itemize}
	\item проанализировать методы обработки событий, возникающих при взаимодействии с USB--устройствами;
	\item проанализировать структуры и функции ядра, предоставляющие информацию о USB--устройствах;
	\item проанализировать способы взаимодействия с графического ускорителя;
	\item разработать алгоритмы и структуру программного обеспечения;
	\item реализовать программное обеспечение;
	\item проанализировать разработанное программное обеспечение.
\end{itemize}

\section{Обработка событий от USB--устройств}

Для обработки событий, возникающих при подключении USB--устройств, необходимо перехватить событие подключения USB--устройства и выполнить необходимую обработку.

Ниже будут рассмотрены существующие подходы к определению возникновения событий подключения USB--устройств и выбран наиболее подходящий для реализации.

\newpage
\subsection{Уведомители ядра}

Ядро Linux содержит механизм, называемый <<уведомителями>> (\texttt{notifiers}) или <<цепочками уведомлений>> (\texttt{notifiers chains}), который позволяет различным подсистемам подписываться на асинхронные события от других подсистем. Цепочки уведомлений в настоящее время активно используется в ядре; существуют цепочки для событий \texttt{hotplug} памяти, изменения политики частоты процессора, события \texttt{USB hotplug}, загрузки и выгрузки модулей \cite{notifications}.

В листинге \ref{lst:notifier_block} представлена структура \texttt{notifier\_block} \cite{notifier_block}.

\begin{lstlisting}[
	caption={Структура \texttt{notifier\_block}},
	label={lst:notifier_block},
	style={c},
	]
struct notifier_block {
	notifier_fn_t notifier_call;
	struct notifier_block __rcu *next;
	int priority;
};
\end{lstlisting}

Данная структура описана в \texttt{/include/linux/notifier.h}. Она содержит указатель на функцию--обработчик уведомления (\texttt{notifier\_call}), указатель на следующий уведомитель (\texttt{next}) и приоритет уведомителя (\texttt{priority}). Уведомители с более высоким значением приоритета выполняются первее.

В листинге \ref{lst:notifier_fn_t} представлен прототип функии \texttt{notifier\_call}.

\begin{lstlisting}[
	caption={Тип \texttt{notifier\_fn\_t}},
	label={lst:notifier_fn_t},
	style={c},
	]
typedef	int (*notifier_fn_t)(struct notifier_block *nb, unsigned long action, void *data);
\end{lstlisting}

Прототип содержит указатель на уведомитель (\texttt{nb}), действие, при котором вызывается функция (\texttt{action}) и данные, которые передаются от действия в обработчик (\texttt{data}).

\newpage
Для регистрации уведомителя для USB--портов используются функции регистрации и удаления уведомителя, представленные в листинге \ref{lst:usb_notifiers}.

\begin{lstlisting}[
	caption={Уведомители на USB--портах},
	label={lst:usb_notifiers},
	style={c},
	]
/* Events from the usb core */
#define USB_DEVICE_ADD		    0x0001
#define USB_DEVICE_REMOVE	0x0002
#define USB_BUS_ADD		           0x0003
#define USB_BUS_REMOVE		   0x0004
extern void usb_register_notify(struct notifier_block *nb);
extern void usb_unregister_notify(struct notifier_block *nb);
\end{lstlisting}

Прототипы и константы для действий описаны в файле\\ \texttt{/include/linux/notifier.h}, а реализации функций --- в файле\\ \texttt{/drivers/usb/core/notify.c}. Соответственно действие \texttt{USB\_DEVICE\_ADD} означает подключение нового устройства, а \texttt{USB\_DEVICE\_REMOVE} --- отключения.

В листинге \ref{lst:notifiers_func_add_remove} представлены функции уведомителя из файла\\ \texttt{/drivers/usb/core/notify.c} для подлючения и отключения USB--устройства.

\begin{lstlisting}[
	caption={Функции уведомители для подлючения и отключения USB--устройства},
	label={lst:notifiers_func_add_remove},
	style={c},
	]
void usb_notify_add_device(struct usb_device *udev)
{
	blocking_notifier_call_chain(&usb_notifier_list, USB_DEVICE_ADD, udev);
}

void usb_notify_remove_device(struct usb_device *udev)
{
	blocking_notifier_call_chain(&usb_notifier_list, USB_DEVICE_REMOVE, udev);
}
\end{lstlisting}

Особенности \texttt{уведомителей}:

\begin{itemize}
	\item возможность привязки своего обработчика к событию;
	\item возможность добавления более чем одного обработчика событий;
	\item возможность использования интерфейса в загружаемом модуле ядра;
\end{itemize}

\subsection{usbmon}

\texttt{usbmon} \cite{usbmon} --- средство ядра Linux, предназначенное для сбора информации о событиях, свазанных с устройствами ввода--вывода, подключенных к USB порту.

\texttt{usbmon} предоставляет информацию о запросах, сделанных драйверами устройств к драйверам хост--контроллера (HCD). Если драйвера хост--контроллера неисправны, то данные, предоставленные \texttt{usbmon}, могут не соответствовать действительным переданным данным.

В настоящее время реализованы два программных интерфейса для взаимодействия с \texttt{usbmon}: текстовый и двоичный. Двоичный интерфейс доступен с помощью символьного устройства в пространстве имен \texttt{/dev}. Текстовый интерфейс устарел, но сохраняется для совместимости.

В листинге \ref{lst:usbmon} представлена структура ответа, полученного после события, случившегося на USB--устройстве (например, подключение к компьютеру).

\begin{lstlisting}[
	caption={Структура \texttt{usbmon\_packet}},
	label={lst:usbmon},
	style={c},
	]
struct usbmon_packet {
	u64 id;			/*  0: URB ID - from submission to callback */
	unsigned char type;	/*  8: Same as text; extensible. */
	unsigned char xfer_type; /*    ISO (0), Intr, Control, Bulk (3) */
	unsigned char epnum;	/*     Endpoint number and transfer direction */
	unsigned char devnum;	/*     Device address */
	u16 busnum;		/* 12: Bus number */
	char flag_setup;	/* 14: Same as text */
	char flag_data;		/* 15: Same as text; Binary zero is OK. */
	s64 ts_sec;		/* 16: gettimeofday */
	s32 ts_usec;		/* 24: gettimeofday */
	int status;		/* 28: */
	unsigned int length;	/* 32: Length of data (submitted or actual) */
	unsigned int len_cap;	/* 36: Delivered length */
	union {			/* 40: */
		unsigned char setup[SETUP_LEN];	/* Only for Control S-type */
		struct iso_rec {		/* Only for ISO */
			int error_count;
			int numdesc;
		} iso;
	} s;
	int interval;		/* 48: Only for Interrupt and ISO */
	int start_frame;	/* 52: For ISO */
	unsigned int xfer_flags; /* 56: copy of URB's transfer_flags */
	unsigned int ndesc;	/* 60: Actual number of ISO descriptors */
};				/* 64 total length */
\end{lstlisting}

Особенности \texttt{usbmon}:

\begin{itemize}
	\item возможность просматривать собранную информацию с помощью специального ПО (например, \texttt{Wireshark});
	\item возможность отслеживать события на одном порте USB или на всех сразу;
	\item отсутствие возможности вызова обработчика при возникновении определенного события.
\end{itemize}

\texttt{usbmon} позволяет отслеживать события, но не позволяет реагировать на них без программной доработки для реализации обработчика.

\subsection{udevadm}

\texttt{udevadm} \cite{udevadm} --- инструмент для управления устройствами \texttt{udev}. Структура \texttt{udev} описана в библиотеке \texttt{libudev} \cite{libudev}, которая не является системной библиотекой Linux. В данной библиотеке представлен программный интерфейс для мониторинга и взаимодействия с локальными устройствами.

При помощи \texttt{udevadm} можно получить полную информацию об устройстве, полученную из его представления в \texttt{sysfs}, чтобы создать корректные правила и обработчики событий для устройства. Кроме того можно получить список событий для устройства, установить наблюдение за ним.

В листинге \ref{lst:udevadm} представлен пример правила обработки событий, задаваемого с помощью \texttt{udevadm}.

\begin{lstlisting}[
	caption={Правила \texttt{udevadm}},
	label={lst:udevadm},
	style={c},
	]
/* rules file */
SUBSYSTEM=="usb", ACTION=="add", ENV{DEVTYPE}=="usb_device",  RUN+="/bin/device_added.sh"
SUBSYSTEM=="usb", ACTION=="remove", ENV{DEVTYPE}=="usb_device", RUN+="/bin/device_removed.sh"

/* device_added.sh */
#!/bin/bash
echo "USB device added at $(date)" >>/tmp/scripts.log

/* device_removed.sh */
#!/bin/bash
echo "USB device removed  at $(date)" >>/tmp/scripts.log
\end{lstlisting}

Особенности \texttt{udevadm}:

\begin{itemize}
	\item возможность привязки своего обработчика к событию;
	\item невозможность использования интерфейса в ядре Linux;
\end{itemize}

\section{USB--устройства в ядре Linux}

\subsection{Структура \texttt{usb\_device}}

Для хранения информации о USB--устройстве в ядре используется структура \texttt{usb\_device}, описанная в \texttt{/inlclude/linux/usb.h} \cite{usb_device}.

Структура \texttt{usb\_device} предствалена в листинге \ref{lst:usb_device}.

\begin{lstlisting}[
	caption={Структура \texttt{usb\_device}},
	label={lst:usb_device},
	style={c},
	]
struct usb_device {
	int		devnum;
	char		devpath[16];
	u32		route;
	enum usb_device_state	state;
	enum usb_device_speed	speed;
	unsigned int		rx_lanes;
	unsigned int		tx_lanes;
	enum usb_ssp_rate	ssp_rate;
	
	struct usb_tt	*tt;
	int		ttport;
	
	unsigned int toggle[2];
	
	struct usb_device *parent;
	struct usb_bus *bus;
	struct usb_host_endpoint ep0;
	
	struct device dev;
	
	struct usb_device_descriptor descriptor;
	struct usb_host_bos *bos;
	struct usb_host_config *config;
	
	struct usb_host_config *actconfig;
	struct usb_host_endpoint *ep_in[16];
	struct usb_host_endpoint *ep_out[16];
	
	char **rawdescriptors;
	
	unsigned short bus_mA;
	u8 portnum;
	u8 level;
	u8 devaddr;
	
	unsigned can_submit:1;
	unsigned persist_enabled:1;
	unsigned have_langid:1;
	unsigned authorized:1;
	unsigned authenticated:1;
	unsigned wusb:1;
	unsigned lpm_capable:1;
	unsigned usb2_hw_lpm_capable:1;
	unsigned usb2_hw_lpm_besl_capable:1;
	unsigned usb2_hw_lpm_enabled:1;
	unsigned usb2_hw_lpm_allowed:1;
	unsigned usb3_lpm_u1_enabled:1;
	unsigned usb3_lpm_u2_enabled:1;
	int string_langid;
	
	/* static strings from the device */
	char *product;
	char *manufacturer;
	char *serial;
	
	struct list_head filelist;
	
	int maxchild;
	
	u32 quirks;
	atomic_t urbnum;
	
	unsigned long active_duration;
	
#ifdef CONFIG_PM
	unsigned long connect_time;
	
	unsigned do_remote_wakeup:1;
	unsigned reset_resume:1;
	unsigned port_is_suspended:1;
#endif
	struct wusb_dev *wusb_dev;
	int slot_id;
	enum usb_device_removable removable;
	struct usb2_lpm_parameters l1_params;
	struct usb3_lpm_parameters u1_params;
	struct usb3_lpm_parameters u2_params;
	unsigned lpm_disable_count;
	
	u16 hub_delay;
	unsigned use_generic_driver:1;
};
\end{lstlisting}

Каждое USB--устройство должно соответствовать спецификации USB--IF \cite{usb_spec}, одним из требований которой является наличие идентификатора поставщика (\texttt{Vendor ID (VID)}) и идентификатор продукта (\texttt{Product ID (PID)}). Эти данные присутствуют в поле \texttt{descriptor} структуры \texttt{usb\_device}. В листинге \ref{lst:usb_device_descriptor} представлена структура дескриптора \texttt{usb\_device\_descriptor}, описанная в \texttt{/include/uapi/linux/usb/ch9.h}.

\begin{lstlisting}[
	caption={Структура \texttt{usb\_device\_descriptor}},
	label={lst:usb_device_descriptor},
	style={c},
	]
/* USB_DT_DEVICE: Device descriptor */
struct usb_device_descriptor {
	__u8  bLength;
	__u8  bDescriptorType;
	
	__le16 bcdUSB;
	__u8  bDeviceClass;
	__u8  bDeviceSubClass;
	__u8  bDeviceProtocol;
	__u8  bMaxPacketSize0;
	__le16 idVendor;
	__le16 idProduct;
	__le16 bcdDevice;
	__u8  iManufacturer;
	__u8  iProduct;
	__u8  iSerialNumber;
	__u8  bNumConfigurations;
} __attribute__ ((packed));

#define USB_DT_DEVICE_SIZE		18
\end{lstlisting}

\newpage
\subsection{Структура \texttt{usb\_device\_id}}

При подключении USB--устройства к компьютеру, оно идентифицируется и идентификационная информация записывается в структуру \texttt{usb\_device\_id} \cite{usb_device_id}.

Структура \texttt{usb\_device\_id} предствалена в листинге \ref{lst:usb_device_id}.

\begin{lstlisting}[
	caption={Структура \texttt{usb\_device\_id}},
	label={lst:usb_device_id},
	style={c},
	]
struct usb_device_id {
	/* which fields to match against? */
	__u16		match_flags;
	
	/* Used for product specific matches; range is inclusive */
	__u16		idVendor;
	__u16		idProduct;
	__u16		bcdDevice_lo;
	__u16		bcdDevice_hi;
	
	/* Used for device class matches */
	__u8		bDeviceClass;
	__u8		bDeviceSubClass;
	__u8		bDeviceProtocol;
	
	/* Used for interface class matches */
	__u8		bInterfaceClass;
	__u8		bInterfaceSubClass;
	__u8		bInterfaceProtocol;
	
	/* Used for vendor-specific interface matches */
	__u8		bInterfaceNumber;
	
	/* not matched against */
	kernel_ulong_t	driver_info
	__attribute__((aligned(sizeof(kernel_ulong_t))));
};
\end{lstlisting}

\newpage
\section{Взаимодействие с графическим ускорителем}

Для взаимодействия с графическим ускорителем используется \texttt{NVML} \cite{nvml}. \texttt{NVML} -- это API для мониторинга и управления состояниями устройств \texttt{NVIDIA} \cite{nvidia}. \texttt{NVML} обеспечивает прямой доступ к запросам и командам утилиты \texttt{nvidia-smi} \cite{nvidia_smi}. 

API \texttt{NVML} поставляется вместе с драйвером дисплея \texttt{NVIDIA}. \texttt{SDK} предоставляет соответствующие заголовки, библиотеки и примеры приложений. 

Все версии \texttt{NVML} имеют обратную совместимость и предназначены для создания приложений сторонних разработчиков.

Для работы с графическим ускорителем, устройство идентифицируется при помощью структуры \texttt{nvmlDevice\_t}. Эти данные можно получить путём вызова функции \texttt{nvmlDeviceGetHandleByIndex}, которая в свою очередь принимает аргумент \texttt{index}, являющийся индексом устройства в списке подключенных GPU устройств. По умолчанию индекс графического ускорителя равен нулю. 

В листинге \ref{lst:nvmlDeviceGetHandleByIndex} представлен прототип функции\\ \texttt{nvmlDeviceGetHandleByIndex}. 

\begin{lstlisting}[
	caption={\texttt{Прототип функции \texttt{nvmlDeviceGetHandleByIndex}}},
	label={lst:nvmlDeviceGetHandleByIndex},
	style={c},
	]
nvmlReturn_t nvmlDeviceGetHandleByIndex(unsigned int index, nvmlDevice_t* device)
\end{lstlisting}

Для получения PCI графического ускорителя при помощи полученной структуры \texttt{nvmlDevice\_t} используется функция \texttt{nvmlDeviceGetPciInfo}.

В листинге \ref{lst:nvmlDeviceGetPciInfo} представлен прототип функции \texttt{nvmlDeviceGetPciInfo}. 

\begin{lstlisting}[
	caption={\texttt{Прототип функции \texttt{nvmlDeviceGetPciInfo}}},
	label={lst:nvmlDeviceGetPciInfo},
	style={c},
	]
nvmlReturn_t nvmlDeviceGetPciInfo(nvmlDevice_t device, nvmlPciInfo_t* pci)
\end{lstlisting}

Для управления состоянием графического ускорителя используется функция \texttt{nvmlDeviceModifyDrainState}. С помощью этой функции графический ускоритель переводится в режим ожидания. В режиме ожидания устройство не принимает никаких новых входящих запросов. С помощью этой же функции графический ускоритель переводится в активное состояние. 

\newpage
В листинге \ref{lst:nvmlDeviceModifyDrainState} представлен прототип функции\\ \texttt{nvmlDeviceModifyDrainState}. 

\begin{lstlisting}[
	caption={\texttt{Прототип функции \texttt{nvmlDeviceModifyDrainState}}},
	label={lst:nvmlDeviceModifyDrainState},
	style={c},
	]
nvmlReturn_t nvmlDeviceModifyDrainState(nvmlPciInfo_t* pciInfo, nvmlEnableState_t newState)
\end{lstlisting}

У всех вышеупомянутых функций возвращаемый тип \texttt{nvmlReturn\_t}. В листинге \ref{lst:nvmlReturn_t} перечислены возможные значения переменной этого типа. 

\begin{lstlisting}[
	caption={\texttt{Возможные значения переменной типа \texttt{nvmlReturn\_t}}},
	label={lst:nvmlReturn_t},
	style={c},
	]
typedef enum nvmlReturn_enum
{
	// cppcheck-suppress *
	NVML_SUCCESS = 0,                        //!< The operation was successful
	NVML_ERROR_UNINITIALIZED = 1,            //!< NVML was not first initialized with nvmlInit()
	NVML_ERROR_INVALID_ARGUMENT = 2,         //!< A supplied argument is invalid
	NVML_ERROR_NOT_SUPPORTED = 3,            //!< The requested operation is not available on target device
	NVML_ERROR_NO_PERMISSION = 4,            //!< The current user does not have permission for operation
	NVML_ERROR_ALREADY_INITIALIZED = 5,      //!< Deprecated: Multiple initializations are now allowed through ref counting
	NVML_ERROR_NOT_FOUND = 6,                //!< A query to find an object was unsuccessful
	NVML_ERROR_INSUFFICIENT_SIZE = 7,        //!< An input argument is not large enough
	NVML_ERROR_INSUFFICIENT_POWER = 8,       //!< A device's external power cables are not properly attached
	NVML_ERROR_DRIVER_NOT_LOADED = 9,        //!< NVIDIA driver is not loaded
	NVML_ERROR_TIMEOUT = 10,                 //!< User provided timeout passed
	NVML_ERROR_IRQ_ISSUE = 11,               //!< NVIDIA Kernel detected an interrupt issue with a GPU
	NVML_ERROR_LIBRARY_NOT_FOUND = 12,       //!< NVML Shared Library couldn't be found or loaded
	NVML_ERROR_FUNCTION_NOT_FOUND = 13,      //!< Local version of NVML doesn't implement this function
	NVML_ERROR_CORRUPTED_INFOROM = 14,       //!< infoROM is corrupted
	NVML_ERROR_GPU_IS_LOST = 15,             //!< The GPU has fallen off the bus or has otherwise become inaccessible
	NVML_ERROR_RESET_REQUIRED = 16,          //!< The GPU requires a reset before it can be used again
	NVML_ERROR_OPERATING_SYSTEM = 17,        //!< The GPU control device has been blocked by the operating system/cgroups
	NVML_ERROR_LIB_RM_VERSION_MISMATCH = 18, //!< RM detects a driver/library version mismatch
	NVML_ERROR_IN_USE = 19,                  //!< An operation cannot be performed because the GPU is currently in use
	NVML_ERROR_MEMORY = 20,                  //!< Insufficient memory
	NVML_ERROR_NO_DATA = 21,                 //!< No data
	NVML_ERROR_VGPU_ECC_NOT_SUPPORTED = 22,  //!< The requested vgpu operation is not available on target device, because ECC is enabled
	NVML_ERROR_INSUFFICIENT_RESOURCES = 23,  //!< Ran out of critical resources, other than memory
	NVML_ERROR_FREQ_NOT_SUPPORTED = 24,  //!< Ran out of critical resources, other than memory
	NVML_ERROR_UNKNOWN = 999                 //!< An internal driver error occurred
} nvmlReturn_t;
\end{lstlisting}

% \section{Передача данных в пространство пользователя}

% Поставленная задача подразумевает передачу данных из пространства ядра пространство пользователя. 

% Способы передачи данных из пространства ядра пространство пользователя:

% \begin{itemize}
%     \item файловая система \texttt{/proc};
%     \item netlink сокеты.
% \end{itemize}

\section{Виртуальная файловая система \texttt{/proc}}

Для организации доступа к разнообразным файловым системам в Unix используется промежуточный слой абстракции -- виртуальная файловая система. Виртуальная файловая система организована как специальный интерфейс. Виртуальная файловая система объявляет API доступа к ней. Это API реализовывается драйверами конкретных файловых систем.

Виртуальная файловая система \texttt{/proc} -- специальный интерфейс, с помощью которого можно мгновенно получить информацию о ядре в пространство пользователя.

В основном дереве каталога \texttt{/proc} в Linux, каждый каталог имеет числовое имя и соответствует процессу, с соответствующим PID. Файлы в этих каталогах соответствуют структуре \texttt{task\_struct}. С помощью команды \texttt{cat /proc/1/cmdline}, можно узнать аргументы запуска процесса с идентификатором равным единице. В дереве \texttt{/proc/sys} отображаются внутренние переменные ядра.

Ядро предоставляет возможность добавить своё дерево в каталог \texttt{/proc}. Внутри ядра объявлена структура \texttt{struct proc\_ops}. Эта структура содержит такие указатели, как указатель на функции чтения и записи в файл. 

В листинге \ref{lst:proc} представлено объявление данной структуры в ядре.

\begin{lstlisting}[
	caption={\texttt{Cтруктура \texttt{struct proc\_ops}}},
	label={lst:proc},
	style={c},
	]
struct proc_ops {
	unsigned int proc_flags;
	int	(*proc_open)(struct inode *, struct file *);
	ssize_t	(*proc_read)(struct file *, char __user *, size_t, loff_t *);
	ssize_t (*proc_read_iter)(struct kiocb *, struct iov_iter *);
	ssize_t	(*proc_write)(struct file *, const char __user *, size_t, loff_t *);
	loff_t	(*proc_lseek)(struct file *, loff_t, int);
	int	(*proc_release)(struct inode *, struct file *);
	__poll_t (*proc_poll)(struct file *, struct poll_table_struct *);
	long	(*proc_ioctl)(struct file *, unsigned int, unsigned long);
	#ifdef CONFIG_COMPAT
	long	(*proc_compat_ioctl)(struct file *, unsigned int, unsigned long);
	#endif
	int	(*proc_mmap)(struct file *, struct vm_area_struct *);
	unsigned long (*proc_get_unmapped_area)(struct file *, unsigned long, unsigned long, unsigned long, unsigned long);
} __randomize_layout;
\end{lstlisting}

С помощью вызова функций \texttt{proc\_mkdir()} и \texttt{proc\_create()} в модуле ядра можно зарегистрировать свои каталоги и файлы в \texttt{/proc}. Функции \texttt{copy\_to\_user()} и \texttt{copy\_from\_user()} реализуют передачу данных из пространства ядра в пространство пользователя и наоборот.

Таким образом, с помощью виртуальной файловой системы \texttt{/proc} можно получать и передавать данные между пространством ядра и пространствм пользовтеля .

% \subsection{Netlink сокеты}
% Netlink представляет  собой способ коммуникации между юзерспейсом и ядром Linux. Коммуникация осуществляется с помощью обычного сокета, с использованием особого протокола — AF\_NETLINK.
% Netlink позволяет взаимодействовать с большим количеством подсистем ядра — интерфейсы, маршрутизация, фильтр сетевых пакетов. Кроме того, можно общаться со своим модулем ядра. Разумеется в последнем должна быть реализована поддержка такого способа коммуникации.
% Каждое сообщение netlink представляет собой заголовок, представленный структурой nlmsghdr, а так же определенного количества байт — «полезной нагрузки» (playload). Данная «нагрузка» может представлять собой какую либо структуру, либо же просто RAW данные. Сообщение, во время доставки, может быть разбито на несколько частей. В таких случаях каждый следующий пакет помечается флагом NLM\_F\_MULTI, а последний флагом NLMSG\_DONE. Для разбора сообщений имеется целый набор макросов, определенный в заголовочных файлах netlink.h и rtnetlink.h

\newpage
\section*{Выводы}

Были рассмотрены методы обработки событий, возникающих при взаимодействии с USB--устройствами. Среди рассмотренных методов был выбран механизм уведомителей, так как он позволяет привязать свой обработчик события, а также реализован на уровне ядра Linux. Были рассмотрены структуры и функции ядра для работы с уведомителями, способ взаимодействия с графическим ускорителем, а также особенности разработки загружаемых модулей ядра. Для передачи данных из пространства ядра в пространство пользователя будет использоваться файловая система \texttt{proc}.