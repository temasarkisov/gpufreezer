\chapter{Технологический раздел}

\section{Выбор языка и среды программирования}

Разработанный модуль ядра написан на языке программирования \texttt{C} \cite{c-language}. Выбор языка программирования \texttt{С} основан на том, что исходный код ядра Linux, все его модули и драйверы написаны на данном языке.

В качестве компилятора выбран \texttt{gcc} \cite{gcc}.

В качестве среды разработки выбрана среда \texttt{Visual Studio Code} \cite{vscode}.

\section{Хранение информации об\\отслеживаемых устройствах}

Для хранения информации об отслеживаемых устройствах объявлена структура \texttt{int\_usb\_device}, которая хранит в себе идентификационные данные устройства (\texttt{PID, VID}) и указатель на элемент списка.

Структура \texttt{int\_usb\_device} представлена в листинге \ref{lst:iud}.

\begin{lstinputlisting}[
	caption={Структура \texttt{int\_usb\_device}},
	label={lst:iud},
	style={c},
	linerange={15-20},
	]{/Users/temasarkisov/MyProjects/BMSTU/gpufreezer/src/gpufreezer.c}
\end{lstinputlisting}

При подключении или удалении устройства, создается экземпляр данной структуры и помещается в список отслеживаемых устройств.

\newpage
В листинге \ref{lst:adiud} представлены функции для работы со списком отслеживаемых устройств.

\begin{lstinputlisting}[
	caption={Функции для работы со списком отслеживаемых устройств},
	label={lst:adiud},
	style={c},
	linerange={89-110},
	]{/Users/temasarkisov/MyProjects/BMSTU/gpufreezer/src/gpufreezer.c}
\end{lstinputlisting}

\section{Идентификация устройства как разрешённого}

Для проверки устройства необходимо проверить его идентификационные данные с данными разрешённых устройств. В листинге \ref{lst:allow} представлены объявление списка разрешённых устройств и функции для идентификации устройства.

\newpage
\begin{lstinputlisting}[
	caption={Функции для идентификации устройств},
	label={lst:allow},
	style={c},
	linerange={24-26, 31-87},
	]{/Users/temasarkisov/MyProjects/BMSTU/gpufreezer/src/gpufreezer.c}
\end{lstinputlisting}

\section{Обработка событий USB--устройства}

При подключении устройство добавляется в список отслеживаемых устройств. После этого происходит проверка на наличие среди отслеживаемых устройств неразрешённых, и, в случае если такие были найдены, происходит изменение режима работы графического ускорителя.

В листинге \ref{lst:ins} представлен обработчик подключения USB--устройства.

\begin{lstinputlisting}[
	caption={Обработчик подключения USB--устройства},
	label={lst:ins},
	style={c},
	linerange={112-132},
	]{/Users/temasarkisov/MyProjects/BMSTU/gpufreezer/src/gpufreezer.c}
\end{lstinputlisting}

При отключении устройство удаляется из списка отслеживаемых устройств. После этого происходит проверка на наличие среди отслеживаемых устройств неразрешённых, и, в случае если такие не были найдены, происходит происходит изменение режима работы графического ускорителя.

В листинге \ref{lst:del} представлен обработчик отключения USB--устройства.

\begin{lstinputlisting}[
	caption={Обработчик отключения USB--устройства},
	label={lst:del},
	style={c},
	linerange={134-153},
	]{/Users/temasarkisov/MyProjects/BMSTU/gpufreezer/src/gpufreezer.c}
\end{lstinputlisting}

\newpage
\section{Регистрация уведомителя\\ для USB--устройств}

В листинге \ref{lst:not} представлено объявление уведомителя и его функции--обработчика.

\begin{lstinputlisting}[
	caption={Уведомитель для USB--устройств},
	label={lst:not},
	style={c},
	linerange={192-214},
	]{/Users/temasarkisov/MyProjects/BMSTU/gpufreezer/src/gpufreezer.c}
\end{lstinputlisting}

В листинге \ref{lst:mod} представлены регистрация и дерегистрация уведомителя при загрузке и удалении модуля ядра соответственно.

\begin{lstinputlisting}[
	caption={Регистрация и дерегистрация уведомителя},
	label={lst:mod},
	style={c},
	linerange={216-249},
	]{/Users/temasarkisov/MyProjects/BMSTU/gpufreezer/src/gpufreezer.c}
\end{lstinputlisting}

\section{Запись данных в виртуальную\\файловую систему \texttt{/proc}}

В листинге \ref{lst:proc_write} представлена запись данных в виртуальную файловую систему \texttt{/proc}. Данные представляют собой состояние работы, в которое нужно перевести графический ускоритель

\begin{lstinputlisting}[
	caption={Запись данных в виртуальную файловую систему \texttt{/proc}},
	label={lst:proc_write},
	style={c},
	linerange={155-190},
	]{/Users/temasarkisov/MyProjects/BMSTU/gpufreezer/src/gpufreezer.c}
\end{lstinputlisting}

\section{Модификация режима работы графического ускорителя}

В листинге \ref{lst:modify_gpu_state_obtain_device} представлено получение данных устройства для дальнейшего взаимодействия с ним.

\begin{lstinputlisting}[
	caption={Получение данных подключенного графического ускорителя},
	label={lst:modify_gpu_state_obtain_device},
	style={c},
	linerange={22-43},
	]{/Users/temasarkisov/MyProjects/BMSTU/gpufreezer/src/modify_gpu_drain_state.c}
\end{lstinputlisting}

В листинге \ref{lst:modify_gpu_state_process} представлен код, предназначенный для обработки изменения в файле \texttt{gpu\_state} в виртуальной файловой системе \texttt{/proc}.

\begin{lstinputlisting}[
	caption={Обработка изменения в файле \texttt{gpu\_state} в виртуальной системе \texttt{/proc}},
	label={lst:modify_gpu_state_process},
	style={c},
	linerange={46-74},
	]{/Users/temasarkisov/MyProjects/BMSTU/gpufreezer/src/modify_gpu_drain_state.c}
\end{lstinputlisting}

\newpage
\section{Примеры работы разработанного ПО}

На рисунках \ref{img:example_1} --- \ref{img:example_5} представлены примеры работы разработанного модуля ядра.

\img{70mm}{example_1}{Работа графического ускорителя до подключения USB--устройства}

\img{30mm}{example_2}{Пример подключения неразрешённого устройства}

\img{6.3mm}{example_3}{Проверка работы графического ускорителя после подключения неразрешённого устройства}

\img{12.5mm}{example_4}{Пример отключения неразрешённого устройства}

\img{70mm}{example_5}{Проверка работы графического ускорителя после отключения неразрешённого устройства}
