\chapter{Конструкторский раздел}

\section{IDEF0 последовательность преобразований}

На рисунках \ref{img:idef0} - \ref{img:idef2} представлена IDEF0 последовательность преобразований.

\img{110mm}{idef0}{Нулевой уровень преобразований}

\img{110mm}{idef1}{Первый уровень преобразований}

\img{110mm}{idef2}{Второй уровень преобразований}

\newpage
\section{Алгоритм отслеживания событий}

Для отслеживания событий подключения и отключения устройств в модуле ядра размещен соответствующий уведомитель, который будет зарегистрирован при загрузке модуля и удалён при его удалении.

В листинге \ref{lst:notify} представлена функция обработки событий.

\begin{lstinputlisting}[
	caption={Обработка событий},
	label={lst:notify},
	style={c},
	linerange={192-214},
	]{/Users/temasarkisov/MyProjects/BMSTU/gpufreezer/src/gpufreezer.c}
\end{lstinputlisting}

Для каждого события есть отдельный обработчик.

\newpage
\section{Алгоритм работы обработчика событий}

На рисунке \ref{img:gpufreezer_algorithm} представлен алгоритм работы обработчика событий .

\img{200mm}{gpufreezer_algorithm}{Алгоритм обработчика событий}

\section{Структура программного обеспечения}

В состав разрабатываемого программного обеспечения входит один загружаемый модуль ядра и одна программа в пространстве пользователя. Загружаемый модуль ядра отслеживает подключенные USB--устройства и, в случае подключения неразрешённого устройства передаёт информацию об этом в виртуальную файловую систему /proc. Программа в пространстве пользователя изменяет режим работы графического ускорителя в зависимости от данных в файле, содержащем данные о режиме работы графического ускорителя в виртуальной файловой системе /proc. Неразрешённым устройством считается устройство, которое не идентифицируется в соответствии со списком допустимых устройств модуля. Список допустимых устройств задается в исходном коде модуля.
