\chapter*{Введение}
\addcontentsline{toc}{chapter}{Введение}

На сегодняшний день вычислительные мощности компьютера это один из самых ценных ресурсов любой компании. Графические ускорители необходимы для реализации любых высокотехнологичных проектов.

В связи с высокой ценой графических вычислительных мощностей компьютера, вероятность взлома и использования в целях злоумышленников компьютеров, оснащёнными такими мощностями повышена. 

Существует много способов для проведения кибератаки, одним из которых является атака при помощи USB--устройства. При проведении такой атаки при подключении устройства к компьютеру можно запустить вредоносный программный код на выполнение, который может как удалить важные данные, так и запустить ПО, паразитирующее графические ускорители \cite{usbmalware}.

Для того, чтобы предотвратить кибератаку, проводимую посредством подключенного USB--устройства, следует строго отслеживать активные устройства в системе. С точки зрения пользовательского опыта, нет возможности запретить подключать новые устройства к компьютеру, так как большинство устройств ввода--вывода подключаются через USB, поэтому мониторинг активных устройств, их анализ и последующее принятие решений являются хорошим способом для избежания кибератаки.

Чтобы обезопасить свои вычислительные мощности, при подключении устройства можно отключать графический ускоритель. Таким образом исключается возможность использования вычислительных мощностей в целях злоумышленников.

Цель работы --- разработать загружаемый модуля ядра Linux для отключения графического ускорителя при подключении USB--устройства.